\documentclass[]{article}
\usepackage{lmodern}
\usepackage{amssymb,amsmath}
\usepackage{ifxetex,ifluatex}
\usepackage{fixltx2e} % provides \textsubscript
\ifnum 0\ifxetex 1\fi\ifluatex 1\fi=0 % if pdftex
  \usepackage[T1]{fontenc}
  \usepackage[utf8]{inputenc}
\else % if luatex or xelatex
  \ifxetex
    \usepackage{mathspec}
  \else
    \usepackage{fontspec}
  \fi
  \defaultfontfeatures{Ligatures=TeX,Scale=MatchLowercase}
\fi
% use upquote if available, for straight quotes in verbatim environments
\IfFileExists{upquote.sty}{\usepackage{upquote}}{}
% use microtype if available
\IfFileExists{microtype.sty}{%
\usepackage{microtype}
\UseMicrotypeSet[protrusion]{basicmath} % disable protrusion for tt fonts
}{}
\usepackage[margin=1in]{geometry}
\usepackage{hyperref}
\hypersetup{unicode=true,
            pdfborder={0 0 0},
            breaklinks=true}
\urlstyle{same}  % don't use monospace font for urls
\usepackage{graphicx}
% grffile has become a legacy package: https://ctan.org/pkg/grffile
\IfFileExists{grffile.sty}{%
\usepackage{grffile}
}{}
\makeatletter
\def\maxwidth{\ifdim\Gin@nat@width>\linewidth\linewidth\else\Gin@nat@width\fi}
\def\maxheight{\ifdim\Gin@nat@height>\textheight\textheight\else\Gin@nat@height\fi}
\makeatother
% Scale images if necessary, so that they will not overflow the page
% margins by default, and it is still possible to overwrite the defaults
% using explicit options in \includegraphics[width, height, ...]{}
\setkeys{Gin}{width=\maxwidth,height=\maxheight,keepaspectratio}
\IfFileExists{parskip.sty}{%
\usepackage{parskip}
}{% else
\setlength{\parindent}{0pt}
\setlength{\parskip}{6pt plus 2pt minus 1pt}
}
\setlength{\emergencystretch}{3em}  % prevent overfull lines
\providecommand{\tightlist}{%
  \setlength{\itemsep}{0pt}\setlength{\parskip}{0pt}}
\setcounter{secnumdepth}{0}
% Redefines (sub)paragraphs to behave more like sections
\ifx\paragraph\undefined\else
\let\oldparagraph\paragraph
\renewcommand{\paragraph}[1]{\oldparagraph{#1}\mbox{}}
\fi
\ifx\subparagraph\undefined\else
\let\oldsubparagraph\subparagraph
\renewcommand{\subparagraph}[1]{\oldsubparagraph{#1}\mbox{}}
\fi

%%% Use protect on footnotes to avoid problems with footnotes in titles
\let\rmarkdownfootnote\footnote%
\def\footnote{\protect\rmarkdownfootnote}

%%% Change title format to be more compact
\usepackage{titling}

% Create subtitle command for use in maketitle
\providecommand{\subtitle}[1]{
  \posttitle{
    \begin{center}\large#1\end{center}
    }
}

\setlength{\droptitle}{-2em}

  \title{}
    \pretitle{\vspace{\droptitle}}
  \posttitle{}
    \author{}
    \preauthor{}\postauthor{}
    \date{}
    \predate{}\postdate{}
  

\begin{document}

\begin{center}\rule{0.5\linewidth}{\linethickness}\end{center}

Report ID: Team8\_2

Overall score: 9/10

Project: 1

\begin{center}\rule{0.5\linewidth}{\linethickness}\end{center}

~

\textbf{1. Please summarize this report in your own words (up to 200
words).} The report focused on analyzing the dataset from the
Student/Teacher Achievement Ratio (STAR) Project. The primary question
considered in the project is whether class type affects the math scores
in the Stanford Achievement Test (SAT) of first-grade students in
Project STAR. The report did not t examine class size effects on a
subset of students (male/female, or black/white) or demographics (inner
cities, urban and rural, etc.). The report stated there is a
considerably small ratio of missing values. These missing values were
assumed to be random, thus, ignored. A one-way ANOVA model was fitted to
the data. Various diagnostics/sensitivity analysis such as Levene's test
for homoscedasticity and quantile-quantile plots analysis was performed
on the model. The model was found to be suitable by the authors. The
authors found a significant difference in first-grade students'
mathematics performance on the SAT between class size, with students in
smaller classes outperforming students in larger classes on average. It
was concluded that due to the experimental design and how randomly
students and teachers are assigned to different class sizes, some causal
statements could be made.

\textbf{2. Are the chosen questions interesting?} Yes. The report
focused on investigating whether class sizes are associated with math
performance while noting other sorts of associations might be possible.
However, these ``other'' associations weren't pursued as they are beyond
the scope of the project.

\textbf{3. Does the report show that the authors understand the data
set?} Yes

\textbf{4. Can the proposed methods answer these questions?} Yes.

\textbf{5. Is the report well-organized and clearly written?} Yes,
however some of the statements are somewhat unclear, e.g., the authors
stated they provide a 99\% Bonferroni confidence interval instead of a
99\% \(\it{family-wise}\) Bonferroni confidence interval.

\textbf{6. Should the report be given extra credit?} I don't think so.

\textbf{7. Please elaborate on your assessments and provide constructive
feedback (no limit).} I believe, overall, the authors of the report did
a good job. The authors followed through with the minimum task required.
However, it would be a lot better if the following are incorporated:

\begin{enumerate}
\def\labelenumi{\arabic{enumi}.}
\item
  It was implicitly assumed that the missing values in the data set
  could be missing at random, but there wasn't any thorough statistical
  analysis carried out to buttress this claim.
\item
  Since these sorts of project reports are geared towards making
  policies, it might augur well that some of the results presented by
  the authors are given in the context of policymaking. This would help
  policymakers understand the effect of whatever decisions they are
  inclined towards.
\item
  The writing style is sometimes unclear or not as precise as possible.
  Thus, the reader is left to fill in the gaps and might end up being
  unintentionally misinformed.
\item
  There seemed to be a lack of consistency in the presentation of
  results, especially the tables. Some results are presented in an
  understandable and clean format while some are presented in raw R
  output. For example, the results of the one-way Anova and the Levene's
  test for homoscedastic. The authors might want to spend some time
  extracting these results from the raw R output and restructuring it
  into a neat and nicely formatted table.
\end{enumerate}

\textbf{8. Please provide questions for authors to address during
presentations.}

\begin{enumerate}
\def\labelenumi{\arabic{enumi}.}
\item
  What is the necessary information that must be provided about the
  aides before the authors can make causal inferences?
\item
  If at all we are interested in investigating further, the effect of
  missing values on our analysis, how do we go about it?
\end{enumerate}


\end{document}
